%	The main skeletal structure
	\documentclass[crop,tikz,convert={outext=.svg,command=\unexpanded{/opt/homebrew/bin/pdf2svg
 \infile\space\outfile}},multi=false]{standalone}
%	\def\pgfsysdriver{pgfsys-tex4ht.def}
%	\documentclass[twocolumn]{report}
	\usepackage{setspace}
	\usepackage{graphicx}
		\DeclareGraphicsExtensions{.pdf,.png,.eps,.ps}
	\usepackage{subcaption}
	\usepackage{lscape}
	\usepackage{pifont}
%	\usepackage{bbding}
	\usepackage{multirow}
	\usepackage{longtable}
	\usepackage[version=4]{mhchem}
	\usepackage{xfrac}
	\usepackage{color}
	\usepackage[colorlinks=true]{hyperref}
	\usepackage{gensymb}
	\usepackage{multicol}
		\setlength{\columnseprule}{0.4pt}
		\setlength{\columnsep}{5mm}
\makeatletter 
\newcounter{reaction} 
%%% >> for article << 
%\renewcommand\thereaction{C\,\arabic{reaction}} 
%%% << for article << 
%%% >> for report and book >> 
\renewcommand\thereaction{C\,\thechapter.\arabic{reaction}} 
\@addtoreset{reaction}{chapter} 
%%% << for report and book << 
\newcommand\reactiontag{\refstepcounter{reaction}\tag{\thereaction}} 
\newcommand\reaction@[2][]{\begin{equation}\ce{#2}% 
\ifx\@empty#1\@empty\else\label{#1}\fi% 
\reactiontag\end{equation}} 
\newcommand\reaction@nonumber[1]{\begin{equation*}\ce{#1}% 
\end{equation*}} 
\newcommand\reaction{\@ifstar{\reaction@nonumber}{\reaction@}} 
\makeatother 


	\usepackage[a4paper]{geometry}
	\usepackage{fullpage}
	\usepackage{fancyhdr}
%		\pagestyle{fancy}
%		\lhead{}
%		\chead{}
%		\rhead{\slshape \rightmark}
%		\fancyhead[LO,RE]{\slshape \leftmark} 
%		\fancyfoot[R]{\thepage} 
%	\renewcommand{\headrulewidth}{0.4pt} 
%	\renewcommand{\footrulewidth}{0.4pt} 
	\usepackage{cite}
	
%	\onehalfspacing
	\renewcommand{\baselinestretch}{1.5}
	
%	Footnote symbols
	\renewcommand{\thefootnote}{\fnsymbol{footnote}}

% Defining the chapter abstract area
%	\newenvironment{abstract}{\rightskip1in}{}

% Allow standard state notation
	\usepackage[varioref=false]{chemstyle}

% Allow for numbered examples
%	\usepackage{theorem,lipsum}
%	\theorembodyfont{\upshape}
%	\newtheorem{example}{Example}[chapter]
%	\newtheorem{question}{Question}[chapter]
%	\newtheorem{exercise}{Exercise}[chapter]
%	\newtheorem{concept}{Key Concept}[chapter]
%%%%%\begin{example}
%%%%%This is an example?
%%%%%\end{example}
%%%%%
%%%%%\begin{question}
%%%%%This is a quesiton?
%%%%%\end{question}

%%%%%FONT STUFF

%\usepackage[defaultfam,extralight,tabular,lining]{montserrat} %% Option 'defaultfam'
%%% only if the base font of the document is to be sans serif
%\usepackage[T1]{fontenc}
%\renewcommand*\oldstylenums[1]{{\fontfamily{Montserrat-TOsF}\selectfont #1}}

\usepackage{arev}
\usepackage[T1]{fontenc}
\usepackage{soul}

%%%%%TIKZ stuff
	\usepackage{tikz}
	\usepackage{pgfplots}
	\usetikzlibrary{decorations.pathmorphing,patterns,arrows,shapes.arrows,quotes,angles,calc}
	\usepgfplotslibrary{fillbetween}
	\tikzset{every picture/.style=remember picture}
		\newcommand{\mathnode}[1]{%
		\mathord{\tikz[baseline=(#1.base), inner sep = 0pt]{\node (#1) {$#1$};}}}

%%%%%% Grey stuff
%Need to define a shitload of greys...
\definecolor{gray1}{RGB}{240,240,240}
\definecolor{gray2}{RGB}{225,225,225}
\definecolor{gray3}{RGB}{210,210,210}
\definecolor{gray4}{RGB}{200,200,200}
\definecolor{gray5}{RGB}{180,180,180}


\begin{document}

\newcommand\angleone{-75}
\newcommand\lengthone{6}
\newcommand\angletwo{-60}
\newcommand\lengthtwo{6}

\begin{tikzpicture}[scale=1]
%\fill [pattern = north east lines] (-0.2,-0.8) rectangle (0,1);
\fill [pattern = north east lines] (-2,0.2) rectangle (5,0);
\draw[thick] (-2,0) -- (5,0);
%\node[circle,ball color=gray1,inner sep=0mm,minimum size=1cm] at (\angleone:\lengthone cm) {};
\draw[densely dotted] (0,0) -- ++(-90:6cm);
\draw[thick] (0,0) -- node[above right]{$L_1$} ++(\angleone:\lengthone cm);
\draw[dashed] (-60:6cm) arc (-60:-100:6cm);
%\draw[<->] (\angleone:0.95*\lengthone cm) arc  (\angleone:-90:0.95*\lengthone cm) node[midway,above]{\tiny $s$}  ;
\draw[<-] (\angleone:2cm) arc  (\angleone:-90:2cm) node[midway,below]{\tiny $\phi_1$}  ;

% Ball two
%\draw[thick,black] (-70:6cm) -- ++(-55:5.5cm);
%\draw[densely dotted] (-70:6cm) -- ++(-90:3cm);
\begin{scope}[shift={(\angleone:\lengthone cm)}]
\draw[thick] (0,0) -- node[above right]{$L_2$} ++(\angletwo:\lengthtwo cm);
\draw[densely dotted] (0,0) -- ++(-90:0.5*\lengthtwo cm);
\draw[<-,blue] (\angletwo:2cm) arc  (\angletwo:-90:2cm) node[midway,below]{\tiny $\phi_2$}  ;
\node[circle,ball color=gray1,inner sep=0mm,minimum size=1cm] at (\angletwo:\lengthtwo cm) {$m_2$};
\end{scope}


\node[circle,ball color=gray1,inner sep=0mm,minimum size=1cm] at (\angleone:\lengthone cm) {$m_1$};
%\draw[<->] (\angleone:0.95*\lengthone cm) arc  (\angleone:-90:0.95*\lengthone cm) node[midway,above]{\tiny $s$}  ;
%\draw[densely dotted,red] (-70:6cm) --++ (20:1cm);
%\draw[densely dotted,red] (0:0cm) --++ (20:1cm);
%\draw[<->,red] (20:0.8cm) -- node[right]{\tiny $L$} ++ (-70:6cm);
%\draw[->,blue,ultra thick] (-70:6cm) -- node[right]{$T$} ++(110:2cm) ;
%\draw[->,blue,ultra thick] (-70:6cm) -- node[left]{$mg$} ++ (-90:1.96cm) ;
%\draw[->,orange, thick] (-70:6cm) -- node[right]{\tiny $mg \cos \phi$} ++ (-70:1.8cm) ;
%\draw[->,orange, thick] (-70:6cm) --  ++ (-160:0.8cm) node[left,below]{\tiny $mg \sin \phi$};
%\draw[<-,red] (-70:7cm) arc  (-70:-90:1cm) node[midway,below]{\tiny $\phi$}  ;
%\node[circle,ball color=gray1,inner sep=0mm,minimum size=1cm,opacity=0.8] at (-70:6cm) {};

\end{tikzpicture}

\end{document}

% Use the following to include the graphics in the chapter
%
%
%\begin{figure}[htbp]
%\begin{center}
%\includegraphics[scale=0.5]{images/fig-ch7_harmonicoscill0.pdf}
%\caption[Caption for list of figures]{Full caption to appear beside the image}
%\label{fig:label}
%\end{center}
%\end{figure}
 